\documentclass[10pt,a4paper]{article}
\usepackage{fontspec}
\defaultfontfeatures{Mapping=tex-text}
\usepackage{xunicode}
\usepackage{xltxtra}
%\setmainfont{???}
\usepackage{polyglossia}
\setdefaultlanguage{english}
\usepackage{amsmath}
\usepackage{amsfonts}
\usepackage{amssymb}
\usepackage{graphicx}
\usepackage[left=2cm,right=2cm,top=2cm,bottom=2cm]{geometry}

\title{Lions User Guide}
\author{Dr. Richard Thompson}

\newcommand{\arrows}[1]{\textless #1\textgreater}

\begin{document}

\maketitle


\section{LIONS Pipeline Architecture}
LIONS is a bioinformatic analysis pipeline which brings together a
 few pieces of software and some home-brewed scripts to annotate a
 paired-end RNAseq library against a reference TE annotation set
 (such as Repeat Masker).

 East Lion proccesses a bam file input, re-aligns it to a genome,
 builds an ab initio assembly using Tophat2. This assembly is then
 proccessed and local read searches are done at the 5' ends to find
 additional transcript start sites and quality control the 5' ends of
 the assembly. The output is a file-type \arrows{library} .lions which
 annotates the intersection between the assembly, a reference gene set
 and repeat set.

 West Lion compiles different .lions files, groups them into biological
 catagories (i.e. Cancer vs. Normal or Treatment vs. Control) and
 compares and analyzes the data to create graphs and meaningful
 interpretation of the data.

\section{LIONS Installation Steps}

\begin{enumerate}
\item Download LIONS core files from github
\item In ./LIONS/resources copy 'example' folder to '\arrows{genomeName}' folder

\item Populate the resource files:
\begin{enumerate}
\item In ./LIONS/resources/\arrows{genomeName}/genome/ add a \arrows{genomeName}.fa genome sequence file
\item In ./LIONS/resources/\arrows{genomeName}/annotation/ add the ucsc annotation file
\item In ./LIONS/resources/\arrows{genomeName}/repeat/ add the ucsc repeatMasker file
\end{enumerate}
\item Set up the project in parameter.ctrl:
\begin{enumerate}
\item Give your project a name and fill out all the parameters as you see fit.
\item In the software list, refer LIONS to the paths for each software on your system. If it's in \$BIN/ already then you can simply type the command
\end{enumerate}
\end{enumerate}

\section{LIONS File Architecture}

% ===================================================================
\subsection{./}
  base directory:  LIONS is a self-contained pipeline and 
  references needed by LIONS from the system are linked
  within this directory.

\begin{description}
\item[./lions.sh \arrows{parameter.ctrl}]
    The master script from which the entire pipeline is ran
    The script reads and procceses all files in input.list
    and all parameters can be controlled from parameter.ctrl
\end{description}

\subsection{./controls}
  Control Files: This folder contains project and system-specific
  parameters for running LIONS. There are three main files which
  need to be set-up, LIONS run parameters, system parameters and
  input RNA-seq libraries.
  
\begin{description}
\item[./controls/parameter.ctrl]
    A bash script which defines global project-specific 
    variables such as Project Name, library input list etc...
    The .sysctrl and .list file are defined here as well

\item[./controls/system.sysctrl]
    A bash script which defines global variables for all LIONS
    scripts. Also defines system-specific variables such as
    System Name, number of CPU cores etc...

\item[./controls/input.list]
    A three column tab-delimited file defining: \newline
    \arrows{Library\_Name}\arrows{Library\_Path\_on\_system}\arrows{Biological\_Grouping}
    Group Naming Convention: 1 = control 2 = experimental 3 = other
\end{description}
    
% ./bin
  Binary: A folder for symbolic links to binaries needed by
  the pipeline and script to initilize the folder. Make sure to set
  the correct commands for the software list in parameters.ctrl for
  your system


\subsection{./projects}
  For each \arrows{Project\_Name} a single folder will be initilized in which
  the data will be organized.

\begin{description}
\item[./projects/\arrows{Project\_Name}]
  The main directory for this project. Each individual library in the
  input will have a folder generated here called \arrows{Library}.

\item[./projects/\arrows{Project\_Name}/logs]
  Folder contains run-specific information such as input file at time
  of run and a copy of the input parameter file at time of run.

\item[./projects/\arrows{Project\_Name}/Analysis(\_RUNID)]
  Not implemented yet. This contains all data analysis for a run
  of LIONS. All graphs and project-wide .lions files are stored here
  
\item[./projects/\arrows{Project\_Name}/\arrows{Library}]
  Library-specific data and primary analysis files.

\item[./projects/\arrows{Project\_Name}/\arrows{Library}/\arrows{Library}.lcsv]
  Raw output file from LIONS containing all possible TE-Exon interaction
  data. This will include initiations, exonizations and terminations
  along with many calculated values about these loci from which
  LIONS will sort initiation events from the others. \arrows{Library}.pc.lcsv
  is the same file with additional information about overlapping protein
  coding genes.

\item[./projects/\arrows{Project\_Name}/\arrows{Library}/\arrows{Library}.lions]
  Initiation only TE-exon data from post-sorting. This file is the 
  complete list of transcripts initiated by TEs in this library. This
  data is passed on to the West Lion protocol to compare TE usage
  between libraries.

\item[./projects/\arrows{Project\_Name}/\arrows{Library}/alignment]
  tophat2-generated alignment and the re-aligned .bam file which
  will be used for analysis. Also contains flagstats and log files.
  Note: Once the alignment is generated it will not be re-generated
  even if you change the alignment parameters in parameter.ctrl. To
  re-make alignments simply start the project with a new name or delete
  these files.

\item[./projects/\arrows{Project\_Name}/\arrows{Library}/assembly]
  cufflinks-generated assembly in 'transcripts.gtf'

\item[./projects/\arrows{Project\_Name}/\arrows{Library}/expression]
  The output from a series of custom scripts 'RNAseqPipeline' which
  will generate wig files and perform RPKM calculations on a series 

\item[./projects/\arrows{Project\_Name}/\arrows{Library}/resources]
  Charlie-foxtrot of library-specific files used to calculate a score
  of parameters in the pipeline.
\end{description}

% ./scripts
  Scripts: all scripts to run lions are held here except for the
  controlling lions.sh script which is in the base folder.
  Check initializeScripts.sh for complete list of scripts
  The main scripts are 

  ./scripts/eastLion.sh
    Alignment, Assembly, Chimeric Detection pipeline

  ./scripts/westLion.sh
    LIONS analysis pipeline

% ./resources
  \arrows{INDEX}: the name of the index set. To be compatible with different
  genome versions, species and gene sets there can be different sets
  of data. The \arrows{INDEX} global variable is set in parameter.ctrl file
  
  ./resources/\arrows{INDEX}/annotation
  \arrows{GENESET} file: A ucsc formatted file containing a gene set to be
  used in the analysis (i.e. look for overlapping genes to transcripts)
  Download from: https://genome.ucsc.edu/cgi-bin/hgTables
  The standard annotation set used was RefSeq 'RefGene' table.

  ./resources/\arrows{INDEX}/genome
  \arrows{INDEX}.fa: The only requisite file here for running LIONS is a fasta
  formatted genome. This could be a symbolic link. LIONS will generate
  the other files neccesary from INDEX.fa. If you have the .bt2 index
  files already generated you can symbolically link them in this folder
  to skip re-generating them.

  ./resources/\arrows{INDEX}/repeat
  \arrows{REPEATMASKER}.ucsc: a ucsc formatted file containing the repeatMasker
  annotation for the genome.
  (Download from UCSC genome browswer or format from RepeatMasker)
  Columns are;
  bin, swScore, milliDiv, milliDel, milliIns, genoName, genoStart,
  genoEnd, genoEnd, genoLeft, strand, repName, repClass, repFamily,
  repStart, repEnd, repLeft, id

% ./software
  Packaged with LIONS is a few bits of software which will set-up your
  system to run the pipeline. Namely setuptools and pysam are the most
  challenging things to install. I found that it's easiest to set-up
  the pipeline using pip and download the package pysam from there.
  Pysam is used to read teh bam files in the python scripts.





\section{LIONS Error Codes}

% LIONS ERROR CODES

\begin{description}
\item[Error 1] internal software error. Check last-run software.
\end{description}

\subsection{Initialization Codes}
\begin{description}


\item[Error 2] Initializal file missing or inaccesible
  A file is missing. Ensure you have a complete version of LIONS and/or
make the missing script readable/exectable

\item[Error 3] A LIONS script is missing
  A script is missing from ./LIONS/scripts/ ; ensure your copy of LIONS is
complete or redownload.

\item[Error 4] Initialization bin missing
  A binary is not found on the system. Configure 
./LIONS/bin/intializeBin.sh for your system

\item[Error 5] A resource file is missing or unreadable
  Checking/initialization of ./LIONS/scripts

\item[Error 6] A python requisite is missing.

\item[Error 7] The input read file (.bam or .fastq) is non-readable or empty
\begin{description}
\item[7A] Bam file error
\item[7B] FastQ file error. Ensure the two files are comma seperated in the input
\end{description}
\end{description}

\subsection{eastLion Error Codes}
%% eastLion Codes --------------------------------

\begin{description}
\item[Error 10] alignment not generated
  An attempt was made to generate an alignment but the output file was
empty at the end of the script

\item[Error 12] wig not generated
  An attempt was made to generate the wig file but the output file wasn't
there after the script ran
\end{description}

\subsection{westLion Error Codes}

%% westLion Codes -----------------------------------
\begin{description}
\item[Error 15] A lions file wasn't generated
  In the run, one of the lions files wasn't generated which means there
was an error. Don't run West Lions pipeline.
\end{description}






\section{LIONS output definitions}

\subsection{Output File-types}

LIONS produces several outputs from different stages of the analysis apart from the standard outputs one would expect (.bam / .gtf).

\begin{description}
\item[`\arrows{library}.lcsv' / `.pc.lcsv']
  These are LIONS CSV files; that is the raw calculations for all major
  numeric operations.
  This includes ALL TE-exon interactions types (Initiation, Exonization and
  Termination). As such there are usually hundreds of thousands of TEs which
  have read fragments joining them to some assembled exon.

  The `.pc.' pre-suffix means the data has been intersected to the input
  set of protein coding genes.

  Use this file for re-calculating ``TE-Initiations" with new parameters.

\item[`\arrows{library}.lion']
  This is the filtered set of TE-exon interactions which have been classified
  as ``TE-Initiations" or TE transcription start sites. This is per-library
  input.

\item[`\arrows{project}.lions'] A merged file of several `.lion' files combining biological groups defined in the `input.list'. A good example of this is merging 10 cancer libraries and 10 normal libraries and outputing only those TE-initiations which are in at least 20\% of Cancer and no Normal libraries. These parameters can be changed in the `paramter.ctrl' input.

\item[`\arrows{project}.rslions'] The `rs' is for Recurrent and Specific TE-initiations only. That is if you   compare the set of libraries 1 (Normal) vs set 2 (Cancer), this contains only those TE-initiations which occur multiple times in Cancer (recurrant) and do not occur in Normal (specific). As defined by `\$cgGroupRecurrence' and `\$cgSpecificity' in the `paramter.ctrl' file.

\item[`\arrows{project}.inv.rslions'] The `.inv.' pre-suffix is simply the \textbf{inverse} of the `.rslion' file. So instead of ``Cancer vs. Normal", ``Normal vs. Cancer". A necessary control if one makes any conclusions based on enrichment/depletion.
\end{description}

\subsection{Output Columns}

\subsubsection{`.lions'}
Most columns should be self-explanatory, some are not.

\begin{description}
\item[transcriptID] Unique identifier for the transcript (isoform). Usually taken
  from the assembly/reference transcriptome

\item[exonRankInTranscript] For each TE-exon interaction combination (row) which exon
  in the `transcriptID' is this row referring to

\item[repeatName] The \arrows{repeat\_name}:\arrows{repeat\_class}:\arrows{repeat\_family} taken from input
  set

\item[coordinates] Useful coordinates for visualizing the interaction. It starts/ends
  in the exon and repeat so when opening in a visualization tool you can see
  the reads spanning this area.

ER\_Interaction: The type of relative intersection in the genome between the exon
  and the repeat. Definitions are relative to the exon. Can be ``Up", ``UpEdge", ``EInside", ``RInside", ``Down", ``DownEdge".

\item[IsExonic] ??

\item[ExonsOverlappingWithRepeat] A list of \arrows{transcriptID:exonRank} which overlap
  the repeat.

\item[ER / DR / DE / DD / Total] A count of the number of TE-Exon sequence fragments
  which join this rows TE and Exon. ER means that one end overlaps the Exon
  and one end overlaps the Repeat exclusively, DD means that both ends of the
  fragment overlap both (dual) exon and repeat ...

\item[Chromosome / EStart / EEnd / EStrand] Start, end and strand of the exon

\item[RStart / REnd / RStrand] Start, end and strand of the repeat

\item[RepeatRank] Relative exon/intron position of the repeat to the contig

\item[UpExonStart / UpExonEnd] Coordinates used for calculating expression of
  genome immediatly adjacent an exon boundary. Useful for quantifying read-
  through or spurious transcriptional events.

\item[UpThread] The number of read 'threads' going upstream of the exon.
  See Manuscript for a figure explaining this.

\item[DownThread] The number of read 'threads' going downstream of the exon.
  See Manuscript for a figure explaining this.

\item[ExonRPKM] RPKM calculation for this exon

\item[ExonMax] The maximum coverage count reached within the exon boundaries. Often
  more reliable measure of expression then RPKM for small exons.

\item[UpExonRPKM / UpExonMax] The expression of the exon immediatly upstream of the
  one this row is referring to. (i.e. Exon 1 expression if the row refers
  to Exon 2). Useful for quantifying the relative increase in expression
  when a TE is acting as an alternative promoter into a downstream exon.

\item[RepeatRPKM / RepeatMaxCoverage]  Expression level within repeat boundaries.

\item[UpstreamRepeatRPKM / UpstreamRepeatMaxCoverage] The expression adjacent to the
  repeat, a test for background expression levels.

\item[RefID] When intersecting to a reference gene set, the gene symbol of any genes
  which intersect the area between the Exon-Repeat coordinates.

\item[RefStrand] Strand of the reference genes defined above

\item[assXref] The strand-relationship between the reference gene and the contig exon
  This accounts for anti-sense long non-coding RNA (as), or transcripts
  which run anti-sense to the reference gene. (s) is sense and (c) means
  complex, often some combination of multiple genes. (u) means it could not
  be determined.

\item[Contribution] An estimate of the promoter contribution of this Repeat TSS to
  the expression of gene in total. Calculated with ExonMax and UpExonMax.

\item[UpCov] Ratio of the coverage adjacent to an exon and the exon expression

\item[UpExonRatio] Ratio of hte expression of the exon and it's upstream exon

\item[ThreadRatio] DownThread / UpThread. Set to [10] if dividing by zero.

\item[RepeatID] A unique Identifer for each Repeat in the genome (left-most
  coordinate). Can repeat and thus be used for determining one repeat
  inititating a trancsript in different assemblies.

\item[LIBRARY] Library from which this repeat-exon interaction was calculated from.
\end{description}


\subsubsection{`.rslions'}
\begin{description}
\item[Normal\_occ] Number of times this TE-initiation was found in the "normal" set of libraries. (Usually set 1)

\item[Cancer\_occ] Number of times this TE-initiation was found in the "cancer" set of libraries. (Usually set 2)

\item[Library] A semi-colon seperated list of the LIBRARY identifiers in which this TE-initiation was found in.
\end{description}  
\end{document}
